\documentclass{article}
\usepackage[utf8]{inputenc}
\usepackage[T1]{fontenc}
\usepackage{lmodern}

\begin{document}

\title{SList - Ohjelmointitekniikka (C++)}
\author{Sami Saada}

\maketitle

\begin{abstract}
Tämän dokumentin olisi tarkoitus valaista kehittämääni testikehystä sekä
toteutustani linkitetystä listasta.
\end{abstract}

\section{Testikehys}

Testikehykseni koostuu kolmesta elementistä: testitapaus, testitapauksia säilövä
säiliö ja näitä säiliöitä ajava suorittaja.

\subsection{TestDriver::TestCase}

Perus testitapaus, jonka testit määritetään run()-metodilla. Luokassa on
apumetodeina liuta tarkistuksia kuormitetuilla parametreilla.

Tämä on oleellisin osa testikehyksestäni.

\section{SList}

Kehitin testikehyksen ensin, joten SListin kehitys sujui kohtuullisesti TDD:tä
hyväksi käyttäen. Jokainen toteutettu ominaisuus on huolella määritelty ja
testattu ennen toteuttamista.

\subsection{Toteutetut ominaisuudet}

\begin{itemize}
    \item Listan ja siihen liittyvien luokkien konstruktorit
    \item front()
    \item iterator
    \item const\_iterator
    \item push\_front()
    \item pop\_front()
    \item insert\_after()
    \item erase\_after()
    \item reverse()
    \item swap()
\end{itemize}

\end{document}
